\documentclass{article}
\usepackage{tableofequations}

\begin{document}

\title{\texttt{tableofequations} package documentation}
\author{Nathanael Farley}
\maketitle

\section{Introduction}
\label{sec:introduction}

This is the documentation for the verb=tableofequations= package. It describes how to make a table of important equations, marked throughout the text with \verb=\markequation= and printed with \verb=\printtableofequations=

\section{Commands}
\label{sec:commands}

\verb=\markequation= takes two arguments, the first being the equation to mark (i.e.\ the math which will appear in the table) and the second is the physical label which appears wherever \verb=\markequation= is placed. The placement of this label is the page referenced in the table of equations, so it's worth checking the label is on the same page as the equation!
\begin{equation}
  \label{eq:1}
  e^{i\pi} - 1 = 0
\end{equation}
\markequation{e^{i\pi} - 1 = 0}{Euler's Equation}

is created by the code
\begin{verbatim}
\begin{equation}
  \label{eq:1}
  e^{i\pi} - 1 = 0
\end{equation}
\markequation{e^{i\pi} - 1 = 0}{Euler's Equation}
\end{verbatim}

The second command defined by this package is the \verb=\printtableofequations=, e.g.\ 
\printtableofequations
is created by
\begin{verbatim}
\printtableofequations
\end{verbatim}

To redefine the title of table equations simply\\ \verb=\renewcommand{\toeheading}{=\emph{\mbox{your name here}}\verb=}=.
\section{Additional Notes}
\label{sec:additional-notes}

There are a couple of caveats of using this package
\begin{itemize}
\item The LaTeX file will need to be compiled twice for any changes to appear in the table of equations
\item The table of equations is actually a collection of \verb=parbox=s arranged like a table. This shouldn't affect anything that I can see, but if you're debugging your file, you should remember that when looking through the errors.
\end{itemize}
\end{document}